\documentclass[a4paper,12pt]{report}

\usepackage[utf8]{inputenc}
\usepackage[french]{babel}
\usepackage[T1]{fontenc}
%\usepackage[scaled]{helvet} % police
\usepackage{lmodern}
\usepackage{layout}
\usepackage[top=2cm, bottom=1.5cm, left=2cm, right=2cm]{geometry}
\usepackage{setspace}
\usepackage{verbatim}
\usepackage{moreverb}
\usepackage{listings}
\usepackage{graphicx}
\usepackage{shorttoc}
\usepackage{xcolor}
\usepackage{currfile}

\setcounter{topnumber}{4}
\setcounter{bottomnumber}{4}
\setcounter{totalnumber}{10}
\renewcommand{\textfraction}{0.15}
\renewcommand{\topfraction}{0.85}
\renewcommand{\bottomfraction}{0.70}
\renewcommand{\floatpagefraction}{0.66}

%\include{chapterStyle}

\makeatletter
\def\thickhrulefill{\leavevmode \leaders \hrule height 1ex \hfill \kern \z@}
\def\@makechapterhead#1{%
  \vspace*{10\p@}%
  {\parindent \z@ \raggedleft \reset@font
            \scshape \@chapapp{} \thechapter
        \par\nobreak
        \interlinepenalty\@M
    \Huge \bfseries #1\par\nobreak
    %\vspace*{1\p@}%
    \hrulefill
    \par\nobreak
    \vskip 100\p@
  }}
\def\@makeschapterhead#1{%
  \vspace*{10\p@}%
  {\parindent \z@ \raggedleft \reset@font
            \scshape \vphantom{\@chapapp{} \thechapter}
        \par\nobreak
        \interlinepenalty\@M
    \Huge \bfseries #1\par\nobreak
    %\vspace*{1\p@}%
    \hrulefill
    \par\nobreak
    \vskip 30\p@
  }}

\renewcommand\thesection{\arabic{section}}
%\renewcommand*\familydefault{\sfdefault} %% Only if the base font of the document is to be sans serif


% Redéfinition de commandes

% Pour avoir les noms de chapitre en 1.1.3 etc...
\renewcommand\thesection{\arabic{chapter}.\arabic{section}}

% Désactiver les alinéas automatiques
\parindent=0cm


% Création du glossaire

\setcounter{tocdepth}{4}

\begin{document}
  \begin{onehalfspace}

  % Page de garde
    \begin{titlepage}
      \begin{center}
        Sébastien Corbin et François-Guillaume Ribreau\\
        CSII 3\ieme année\\
    		Le 21 Septembre 2011\\
      \end{center}
      \hrulefill
      \vspace{7cm}
      \begin{center}
        \LARGE \textbf{Cahier des charges}\\
        \vspace{3cm}
        \normalsize Génie Logiciel Embarqué
      \end{center}

      \vspace{9,5cm}

      \begin{center}
      \line(1,0){250}
      \end{center}

      \begin{center}
      \tiny{\currfilename}
      \end{center}


    \end{titlepage}
    \clearpage

	%%%%%%%%%%%%%%%%%%%%
	\chapter*{Expression des besoins}
	%%%%%%%%%%%%%%%%%%%%
	
	\paragraph*{}
	Il s'agit d'une application se basant sur la géolocalisation pour diffuser des messages aux utilisateurs. Ces messages devront s'inscrire dans une optique de partage en temps réel.
	
	\paragraph*{}
	Cette application se basera sur une plateforme qui devra permettre aux utilisateurs d'ajouter des applications géolocalisées sur leur mobile.		

	\paragraph*{}
	L'application se veut simpliste, avec un système d'envoi et d'affichage de messages, ainsi que la gestion de son profil.
	
	%%%%%%%%%%%%%%%%%%%%
	\chapter*{Etude de faisabilité}
	%%%%%%%%%%%%%%%%%%%%
	
	%%%%%%%%%%%%%%%%%%%%
	\section*{Terminaux mobiles}
	
	Il existe actuellement 5 principaux systèmes d'exploitation mobile sur le marché :
	\begin{itemize}
		\item Android de Google Inc.
		\item BlackBerry OS de RIM
		\item iOS d'Apple
		\item Symbian OS de Symbian Foundation
		\item Windows Phone de Microsoft
	\end{itemize}
	
	\paragraph*{}
	Pour les objectifs de ce cours (informatique embarquée), l'application se voudra native. Mais du fait que tous ces systèmes d'exploitation exploitent un kit de développement différent, il faudra n'en choisir qu'un par contrainte temporelle.
	
	\paragraph*{}
	Au mois d'octobre 2011, les parts du marché mondial des systèmes d'exploitation mobiles (smartphones et tablettes) se répartissaient de la manière suivante :
	\begin{description}
		\item[Apple iOS] 61,4\%
		\item[Google Android] 18,9\%
		\item[Java ME] 12,8\%
		\item[Symbian] 3,5\%
		\item[RIM] 2,5\%
		\item[Autres] 0,7\%
	\end{description}
	\emph{Source : Net Applications}
	
	\paragraph*{}
	De par ces statistiques et le matériel à disposition des élèves (ordinateurs Mac et smartphones Apple), le choix a été porté sur le développement d'une application native iPhone sous iOS.
	
	\paragraph*{}
	Les smartphones d'Apple sont entre autres équipés d'un système de géolocalisation par satellite (GPS) répondant aux besoins de l'application.

	%%%%%%%%%%%%%%%%%%%%
	\section*{Serveur}
	
	\paragraph*{}
	Compte tenu de l'expérience de l'équipe dans le développement d'applications sous iOS, le serveur devra être mis en place et développé rapidement pour pouvoir faire des tests avec l'application mobile au plus tôt dans le cycle de développement.

	\paragraph*{}
	Le serveur devra répondre aux requêtes des terminaux mobiles pour donner et enregistrer les messages de ceux-ci, le besoin de rapidité sera un élément décisif.
	
	\paragraph*{}
	Ainsi, les technologies gratuites et open-source traditionnelles (telles que LAMP) seront écartées par souci de rapidité. En place, un serveur Node.JS et une base données MongoDB seront utilisées.
	Le SGBD MongoDB prend en charge nativement la géolocalisation autant dans l’indexation que dans le requêtage et ses temps d'accès sont très faibles.
	Le serveur Node.JS reprend le très connu langage Javascript côté serveur pour répondre à des événements très rapidement.
	
	\paragraph*{}
	Le serveur devra exécuter un système d'exploitation Unix pour permettre l'utilisation de ces technologies.
	
	
	%%%%%%%%%%%%%%%%%%%%
	\chapter*{Spécifications}
	%%%%%%%%%%%%%%%%%%%%

	%%%%%%%%%%%%%%%%%%%%
	\section*{Application}
	
	\begin{description}
	\item[Plateforme :] Apple (iPhone, iPad)
	\item[Language :] Objective-C
	\item[Dénomination :] Application de notifications géolocalisées
	\item[Nom de code :] GeoBbs
	\item[Utilisation des capacités matérielles :] GPS
	\item[Languages :] Objective-C (client mobile) / JavaScript (côté serveur)
	\end{description}

	%%%%%%%%%%%%%%%%%%%%
	\section*{Serveur}

	\begin{description}
	\item[Base de données :] MongoDB
	\item[Serveur métier :] NodeJS
	\end{description}

	\newpage

	\end{onehalfspace}
\end{document}

