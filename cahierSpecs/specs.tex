\documentclass[a4paper,12pt]{report}

\usepackage[utf8]{inputenc}
\usepackage[french]{babel}
\usepackage[T1]{fontenc}
%\usepackage[scaled]{helvet} % police
\usepackage{lmodern}
\usepackage{layout}
\usepackage[top=2cm, bottom=1.5cm, left=2cm, right=2cm]{geometry}
\usepackage{setspace}
\usepackage{verbatim}
\usepackage{moreverb}
\usepackage{listings}
\usepackage{graphicx}
\usepackage{shorttoc}
\usepackage{xcolor}
\usepackage{currfile}
\usepackage{hyperref}

\setcounter{topnumber}{4}
\setcounter{bottomnumber}{4}
\setcounter{totalnumber}{10}
\renewcommand{\textfraction}{0.15}
\renewcommand{\topfraction}{0.85}
\renewcommand{\bottomfraction}{0.70}
\renewcommand{\floatpagefraction}{0.66}

%\include{chapterStyle}

\makeatletter
\def\thickhrulefill{\leavevmode \leaders \hrule height 1ex \hfill \kern \z@}
\def\@makechapterhead#1{%
  \vspace*{10\p@}%
  {\parindent \z@ \raggedleft \reset@font
            \scshape \@chapapp{} \thechapter
        \par\nobreak
        \interlinepenalty\@M
    \Huge \bfseries #1\par\nobreak
    %\vspace*{1\p@}%
    \hrulefill
    \par\nobreak
    \vskip 100\p@
  }}
\def\@makeschapterhead#1{%
  \vspace*{10\p@}%
  {\parindent \z@ \raggedleft \reset@font
            \scshape \vphantom{\@chapapp{} \thechapter}
        \par\nobreak
        \interlinepenalty\@M
    \Huge \bfseries #1\par\nobreak
    %\vspace*{1\p@}%
    \hrulefill
    \par\nobreak
    \vskip 30\p@
  }}

\renewcommand\thesection{\arabic{section}}
%\renewcommand*\familydefault{\sfdefault} %% Only if the base font of the document is to be sans serif


% Redéfinition de commandes

% Pour avoir les noms de chapitre en 1.1.3 etc...
\renewcommand\thesection{\arabic{chapter}.\arabic{section}}

% Désactiver les alinéas automatiques
\parindent=0cm


% Création du glossaire

\setcounter{tocdepth}{4}

\begin{document}
  \begin{onehalfspace}

  % Page de garde
    \begin{titlepage}
      \begin{center}
        Sébastien Corbin et François-Guillaume Ribreau\\
        CSII 2\ieme année\\
        Le 21 Septembre 2011\\
      \end{center}
      \hrulefill
      \vspace{7cm}
      \begin{center}
        \LARGE \textbf{Cahier des spécifications}\\
        \vspace{3cm}
        \normalsize Génie Logiciel Embarqué
      \end{center}

      \vspace{9,5cm}

      \begin{center}
      \line(1,0){250}
      \end{center}

      \begin{center}
      \tiny{\currfilename}
      \end{center}


    \end{titlepage}
    \clearpage

  \thispagestyle{empty}
  \setcounter{page}{0}
  \clearpage

\chapter*{Ce qu'il faut traiter}
Intro
  Les os de l'embarqué
  ->(liste et ce qu'il font chacun)

  Notion d'entrée/sortie

  Les Env. de dev (avec particularité du remote debugging)

  Résumé des différentes méthodes (agile, xp etc..., tdd etc...)

Projet
  -> Equipe
  -> Cahier des charges

Choix de la méthodo:
  -> Méthode Agile
  -> + TDD

Languages:
Objective-C / JavaScript

Libraries:
ExpressJS (serveur http)
Mongoose http://mongoosejs.com/ (mongodb)

La sécuration est faites par oAuth

\chapter*{Matériel existant}
iPhone

MacBookPro 13 et 15 pouces

\chapter*{Analyse des librairies}

\href{http://developer.apple.com/library/ios/documentation/iPhone/Conceptual/iPhoneOSProgrammingGuide/index.html}{Guide de programmation iOs}

Géolocalisation:

\href{http://developer.apple.com/library/ios/documentation/UserExperience/Conceptual/LocationAwarenessPG/Introduction/Introduction.html}{Introduction au développement relatif à la géolocalisation}

\chapter*{Interface utilisateur}
  \section*{Accueil}
  \begin{description}
    \item[Cas 1 :] GeoBbs est lié à Twitter, Foursquare, Facebook.

    Affichage des derniers Tweets/Check-in (avec algorithme d’importance) de ses amis dans le secteur où se trouve l’utilisateur

    \item[Cas 2 :] Aucun compte lié

    Affichage des dernières notifications envoyées dans le secteur où se trouve l’utilisateur.
    L’UI de la home ressemble à Foursquare:
      \begin{description}
      \item[Liste] Une liste des notifications
      \item[Menu en bas] Liste | Notifier | Réalité augmentée
      \end{description}
  \end{description}

  \end{onehalfspace}
\end{document}
